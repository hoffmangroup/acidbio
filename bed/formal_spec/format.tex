\documentclass[12pt]{article}

\usepackage{amsmath}
\usepackage{amssymb}
\usepackage{amsfonts}
\usepackage{fancyvrb}
\usepackage{lstlisting}

\usepackage[margin=1in]{geometry}


\begin{document}

\section{The BED Format Specification}

\subsection{An example}

\subsection{Terminologies and Concepts}

\subsubsection{Character set restrictions}

\subsection{Header lines}

Header lines are used to define a custom track for use in the Genome Browser.

\subsubsection{Browser lines}

Browser lines configure the overall display of the Genome Browser window when your annotation file is uploaded. Each line defines one display attribute. (Copied from https://genome.ucsc.edu/goldenPath/help/customTrack.html)

\begin{tabular}{ |c|c| }
\hline
Attribute Name & Description \\
\hline
position <position> & (Description) Accepted format: chromosone:start-end.
\begin{Verbatim}[frame=single]
  (Some regex here)
\end{Verbatim}
\hline
hide all & (text)
\hline
\end{tabular}

\subsubsection{Track lines}

\subsection{BED fields}
\begin{tabular}{c c c c c}
Col & Field & Type & Regexp/Range & Brief Description \\
\hline
1 & chrom & String & (Some regex) & (Some Description) \\
2 & chromStart & Int & [0, N) & (Some Description) \\
3 & chromEnd & Int & [1, N) & (Some Description) \\
4 & name & String & (Some regex) & (Some Description)
\hline
\end{tabular}

\begin{enumerate}
  \item chrom:
  \item chromStart:
  \item chromEnd:
\end{enumerate}
\section{Recommended Practice for the BED Format}

\end{document}
